
\documentclass[12pt,letterpaper]{article}


%---------------------------------------------
%
%       THIS IS THE HIGHLIGHTED VERSION.
%       For the final version, carefully remove:
%       '\protect'
%       '\hl{' (don't forget the '}'
%
%---------------------------------------------



%Packages
\usepackage{xcolor}
\usepackage{color,soul}
\usepackage{changepage}
\usepackage{pdflscape}
\usepackage{fixltx2e}
\usepackage{textcomp}
\usepackage{fullpage}
\usepackage{natbib}
\usepackage{float}
\usepackage{latexsym}
\usepackage{url}
\usepackage{epsfig}
\usepackage{graphicx}
\usepackage{amssymb}
\usepackage{amsmath}
\usepackage{bm}
\usepackage{array}
\usepackage[version=3]{mhchem}
\usepackage{ifthen}
\usepackage{caption}
\usepackage{hyperref}
\usepackage{amsthm}
\usepackage{amstext}
\usepackage{enumerate}
\usepackage[osf]{mathpazo}
\usepackage{dcolumn}
\usepackage{lineno}
\usepackage{longtable}
\pagenumbering{arabic}


%Pagination style and stuff
\linespread{2}
\raggedright
\setlength{\parindent}{0.5in}
%\setcounter{secnumdepth}{0} 

\begin{document}
\vspace*{0.35in}

%Running head
\begin{flushright}
%Version dated: \today
\end{flushright}
\bigskip
\noindent Running head: Missing data and topology in Total Evidence method

\bigskip
\medskip
\begin{center}

\noindent{\Large \bf Effects of missing data on topological inference using a Total Evidence approach}

\bigskip

\noindent {\normalsize \sc Thomas Guillerme$^a$$^,$$^b$$^*$, and Natalie Cooper$^a$$^,$$^b$$^,$$^1$}\\
\noindent {\small \it 
$^a$School of Natural Sciences, Trinity College Dublin, Dublin 2, Ireland.\\
$^b$Trinity Centre for Biodiversity Research, Trinity College Dublin, Dublin 2, Ireland.\\
$^*$Corresponding author. Zoology Building, Trinity College Dublin, Dublin 2, Ireland; E-mail: guillert@tcd.ie; Fax: +353 1 6778094; Tel: +353 1 896 2571.\\
$^1$Present address: Department of Life Sciences, Natural History Museum, Cromwell Road, London, SW7 5BD, UK. E-mail: nhcooper12@gmail.com}\\
\end{center}

\vspace{1in}

%Line numbering
\modulolinenumbers[1]
\linenumbers

\newpage
\section* {Abstract}
To fully understand macroevolutionary patterns and processes, we need to include both extant and extinct species in our models.
This requires phylogenetic trees with both living and fossil taxa at the tips.
One way to infer such phylogenies is the Total Evidence approach which uses molecular data from living taxa and morphological data from living and fossil taxa.

Although the Total Evidence approach is very promising, it requires a great deal of data that can be hard to collect.
Therefore this method is likely to suffer from missing data issues that may affect its ability to infer correct phylogenies.

Here we use simulations to assess the effects of missing data on tree topologies inferred from Total Evidence matrices.
We investigate three major factors that directly affect the completeness and the size of the morphological part of the matrix: the proportion of living taxa with no morphological data, the amount of missing data in the fossil record, and the overall number of morphological characters in the matrix.
We infer phylogenies from complete matrices and from matrices with various amounts of missing data, and then compare missing data topologies to the ``best'' tree topology inferred using the complete matrix.

We find that the number of living taxa with morphological characters and the overall number of morphological characters in the matrix, are more important than the amount of missing data in the fossil record for recovering the ``best'' tree topology.
Therefore, we suggest that sampling effort should be focused on morphological data collection for living species to increase the accuracy of topological inference in a Total Evidence framework.
Additionally, we find that Bayesian methods consistently outperform other tree inference methods.
We therefore recommend using Bayesian consensus trees to fix the tree topology prior to further analyses.

\bigskip
\noindent
\textbf{Keywords:} morphological characters, Bayesian, Maximum Likelihood, topology, fossil, living.

%---------------------------------------------
%
%       INTRODUCTION
%
%---------------------------------------------

\newpage
\section{Introduction}
Although most species that have ever lived are now extinct \citep{novacek1992ext,raup1993extinction}, the \hl{many large-scale macroevolutionary studies focus solely on living species} (e.g. \citealp{meredithimpacts2011,jetzthe2012}).
Ignoring fossil taxa may lead to misinterpretation of macroevolutionary patterns and processes such as the timing of diversification events \citep[e.g.][]{pyrondivergence2011}, relationships among lineages \citep[e.g.][]{manosphylogeny2007} or niche occupancy \citep[e.g.][]{pearmanniche2008}.
This has led to increasing consensus among evolutionary biologists that fossil taxa should be included in macroevolutionary studies \citep{jacksonwhat2006,quentaldiversity2010,dietlconservation2011,slaterunifying2013,fritzdiversity2013}.
To do this, however, we need to be able to place living and fossil taxa into the same phylogenies; a task that remains difficult despite recent methodological developments \citep[e.g.][]{pyrondivergence2011,ronquista2012,BEASTmaster}.

Up to now, three main approaches have been used to place both living and fossil taxa into phylogenies.
\hl{These approaches differ mainly in how they treat the fossil taxa and it's data.
One can use the fossil as tips or as nodes in the phylogeny and use it's age only, it's morphology only or it's age and morphology jointly.}
Classical cladistic methods use matrices containing morphological data from both living and fossil taxa and treat each taxon as a tip in the phylogeny. Relationships among the taxa are then inferred using optimality criteria such as maximum parsimony \citep{Hennig1966,felsenstein2004}.
This approach is commonly used by paleontologists but it ignores the additional molecular data available from living species and does not allow use of probabilistic methods for dealing with phylogenetic uncertainty.
Neontologists, on the other hand, more commonly use probabilistic approaches (e.g. Maximum Likelihood or Bayesian methods) based on matrices containing only molecular data from living species.
\hl{Because fossil taxa do not usually have available DNA, only fossils' occurrence dates are used to time calibrate phylogenies} \citep{zuckerkandl1965}.
There have been great improvements in the theory and application of these two approaches \citep[e.g.][]{bapsta2013,stadlerdating2013,heaththe2013} as well as much debate about the ``best'' approach to use \citep[e.g.][]{spencerefficacy2013,wrightbayesian2014}.
Neither approach, however, uses all the available data.

A final approach, known as the Total Evidence method, uses matrices containing molecular data from living taxa and morphological data from both living and fossil taxa \citep{eernissetaxonomic1993}.
This approach treats every taxon as a tip in the phylogeny, uses the occurrence age of the fossils to time calibrate the phylogeny \citep[known as tip-dating;][]{ronquista2012}, and allows the use of probabilistic methods for estimating phylogenetic uncertainty \citep{ronquista2012}.
\hl{Total Evidence methods is becoming an increasingly popular way of adding fossil taxa to phylogenies} \citep[e.g.][]{pyrondivergence2011,ronquista2012,schragocombining2013,slaterphylogenetic2013,beckancient2014,Arcila2015131}.
Although the Total Evidence approach seems very promising, there is one big drawback in using this approach: it requires both molecular and morphological data, both of which can be difficult (or impossible) to collect for every living and fossil taxon in the tree.
Morphological data for living taxa are rarely collected when molecular data are available (e.g. \citealp{O'Leary08022013} \textit{vs.} \citealp{meredithimpacts2011}), and for fossil taxa, data can only be collected from features preserved in the fossil record.
For example, in vertebrates, the hardest parts of the skeleton are more often preserved than soft parts \citep{sansomfossilization2013}; and molecular data is (nearly) always unavailable.
Therefore \hl{Total Evidence matrices are likely to contain a large proportions of missing data} that may affect the method's ability to infer correct topologies, branch lengths and support values \citep{salamin2003}. 

Although missing data does not appear be a major problem in molecular and morphological matrices separately \citep[\hl{as long as enough data overlaps in each case, and missing data are not phylogeneticaly biased;}][]{wiensmissing2003,Wiens01102005,wiensmissing2006,wiensmissing2008,lemmonthe2009,Sanderson22072011,rouresite-specific2011,pattinsonphylogeny2014}, it may become more of an issue in Total Evidence matrices containing both molecular and morphological data for living and fossil taxa.
This may be particularly problematic as fossil taxa (generally) do not have molecular data, resulting in a large section of missing data in Total Evidence matrices.
\hl{Until now, only few attempts have been made to study study the impact of this missing data issue on phylogenetic inference in a Total Evidence framework \protect\citep[\hl{i.e. using combined molecular and morphological data}; e.g.][]{Wiens01102005,pattinsonphylogeny2014}.}
\hl{For example, \protect\cite{Wiens01102005} }\hl{only assess the effect of missing data by comparing the node support between the complete dataset that contains missing data and sub-datasets with no missing data.}
\hl{Furthermore, \protect\cite{pattinsonphylogeny2014} }\hl{assess the effect of removing both molecular and morphological data (``artificial extinction'') on the topology.}
\hl{As for morphological and molecular data separately, these studies shows that missing data is not a major problem and should not be an obstacle to combine both living and fossil species in the same phylogenies.}
\hl{These studies, however, are limited by their empirical-only approach \protect\citep{Wiens01102005} }\hl{and the scheme of generating missing data \protect\citep{pattinsonphylogeny2014}.}
\hl{Also both studies focus on the palaeontological aspect of the question (i.e. the fossil data).}

\hl{In this study, we propose a theoretical assessment of the effect of missing data in the Total Evidence method by thoroughly eliminating living or fossil data or all data overall.
We test the effect of missing data by measuring two crucial aspects of topology in both Maximum Likelihood and Bayesian Inference methods: (i) the conservation of clades and (ii) the displacement of wild-card taxa.}
We focus on the effect of missing data on our ability to recover a ``best'' tree topology because it is a crucial aspect of a phylogeny in many macroevolutionary studies, for example when trying to elucidate the evolutionary relationships among species \citep[e.g.][]{meredithimpacts2011,jetzthe2012}, or for studying evolutionary transitions \citep[e.g.][]{friedmanexplosive2010}.
Although branch length estimation is also important \citep[namely for timing extinction and/or speciation events; e.g.][]{ronquista2012}, we do not consider branch lengths in this study.
This is partially due to difficulties with simulating branch lengths and topology simultaneously, \hl{but also because previous studies have already empirically assessed the effect of the Total Evidence method on branch length variation but using topological constraints} \citep{ronquista2012,schragocombining2013,slaterphylogenetic2013,beckancient2014}.
Thus understanding the sensitivity of topology to missing data is important for assessing the accuracy of tree estimation in the Total Evidence framework. To our knowledge, this question has never been formally assessed.

Here we use a simulation approach to assess the effect of missing data on tree topologies inferred from Total Evidence matrices.
Since the molecular part of a Total Evidence matrix acts like a ``classical" molecular matrix containing only the living taxa \citep{ronquista2012}, the effect of missing data on such matrices is well known \citep{wiensmissing2006,wiensmissing2008,lemmonthe2009,rouresite-specific2011}.
Therefore, we focus only on missing data in the morphological part of the matrix.
We investigate three major parameters that directly affect the completeness and size of the morphological part of the matrix, and reflect empirical biases in data availability: (i) the proportion of living taxa with no morphological data; (ii) the proportion of missing data in the fossil taxa; and (iii) the proportion of missing morphological characters for both living and fossil taxa in the matrix (i.e. the size of the matrix).
\hl{We remove data from a Total Evidence matrix by changing the values of these three parameters and then assess how this affects the topology of trees inferred using Maximum Likelihood and Bayesian methods by using two different topological metrics, based on the Robinson-Foulds }\citep{RF1981}\hl{ and Triplets }\citep{critchlowthe1996}\hl{ distances.
Thus, in the present study, we propose a new approach on the missing-data question focus on both palaeontological and neontological data (rather than just palaeontological data), using only simulated data and two distinct topological metrics.}
We find that minimizing the number of living taxa with no morphological data and the number of missing morphological characters improves the ability of Total Evidence methods to recover the ``best'' tree topology more so than minimizing the amount of missing data in the fossil record.
Additionally, we find that the ability of Total Evidence methods to recover the ``best'' tree topology is increased when using Bayesian methods.


%---------------------------------------------
%
%       METHODS
%
%---------------------------------------------
 
\section{Materials and Methods}
To explore how missing data in \hl{the morphological partition of Total Evidence matrices} influences tree topology, \hl{we used the following protocol (Figure 1):}
\begin{enumerate}
\item{Generating the matrix:} \label{step:generate_matrix} \\
We randomly generated a birth-death tree (hereafter called the ``true'' tree) and used it to simulate a matrix containing both molecular and morphological data for living and fossil taxa (hereafter called the ``complete'' matrix).
\item{Removing data:} \label{step:remove_data} \\
We removed data from the morphological part of the ``complete'' matrix to simulate the effects of missing data by modifying three parameters (i) the proportion of living taxa with no morphological data ($M_{L}$), (ii) the proportion of missing data in the fossil taxa ($M_{F}$) and (iii) the number of morphological characters ($N_{C}$). We call the resulting 125 matrices ``missing-data'' matrices.
\item{Estimating phylogenies:} \label{step:build_phylo} \\
We inferred phylogenetic trees from the ``complete'' matrix and from the 125 ``missing-data'' matrices resulting in one tree generated from a matrix containing no missing data (hereafter called the ``best'' tree) and 125 trees inferred from the matrices with missing morphological data (hereafter called the ``missing-data'' trees). Phylogenies were inferred via both Maximum Likelihood and Bayesian approaches.
\item{Comparing topologies:} \label{step:compare_topo} \\
We compared the ``best'' tree to the ``missing-data'' trees to assess the influence of each parameter ($M_{L}$, $M_{F}$, $N_{C}$) and their interactions on the topologies of our phylogenies
\end{enumerate}
We repeated these four steps 50 times to account for variation in our random parameters in the simulations.

%\begin{figure}[h]
%\caption{{\bf Protocol outline.}
%(1) We randomly generated a birth-death tree (the ``true'' tree) and used it to simulate a matrix with no missing data (the ``complete'' matrix).
%(2) We removed data from the morphological part of the ``complete'' matrix resulting in 125 ``missing-data'' matrices.
%(3) We built phylogenetic trees from each matrix using both Maximum Likelihood and Bayesian methods.
%(4) We compared the ``missing-data'' trees to the ``best'' tree.
%We repeated these four steps 50 times.}
%\label{Fig_Outline}
%\end{figure}

%---------------------------------------------
%       1-Generating the matrix
%---------------------------------------------

\subsection{Generating the matrix}
\label{Generating_the_matrix}
First we randomly generated a ``true'' tree of 50 taxa in R v. 3.0.2 \citep{R302} using the package diversitree v. 0.9-6 \citep{fitzjohndiversitree2012}.
We generated the tree using a birth death process by sampling speciation ($\lambda$) and extinction ($\mu$) rates from a uniform distribution (bounded between 0 and 1) but maintaining $\lambda$ $>$ $\mu$ \citep{paradistime-dependent2011}.
Empirical Total Evidence matrices vary in whether they have more fossil than living taxa or vice versa.
For example, fossil taxa make up 88\% \citep{beckancient2014}, 58\% \citep{schragocombining2013}, 48\% \citep{pyrondivergence2011}, 31\% \citep{ronquista2012} and 31\% \citep{slaterphylogenetic2013} of taxa in various studies.
To avoid biasing our simulations towards either living or fossil taxa and to make each simulation comparable, we implemented a rejection sampling algorithm to select only trees with 25 living and 25 fossil taxa.
The fossil taxa were considered as unique tips at the end of extinct lineages.
We then added an outgroup to the tree, using the mean branch length of the tree to separate the outgroup from the rest of the taxa, and with the branch length leading to the outgroup set as the sum of the mean branch length and the longest root-to-tip length of the tree.

Next, we generated a molecular and a morphological matrix from the ``true'' tree.
The molecular matrix was \hl{simulated} from the ``true'' tree using the R package phyclust v. 0.1-14 \citep{chen2011}.
The matrix contained 1000 character sites for 51 taxa and was generated using the seqgen algorithm \citep{ranbaut1997seqgen} and using the HKY model \citep{HKY85} with random base frequencies (sampled from a uniform probability distribution bounded between 0 and 1 with the total frequency for the four bases equal to 1) and transition/transversion rate of two \citep{douadycomparison2003}.
\hl{The substitution rates were selected from a gamma distribution with an ($\alpha$) shape of 0.5} \citep{yangamong-site1996}.
\hl{In practice, such a value of $\alpha$ $<$ 1 decreases the number of sites with high substitution rates, thus reducing homoplasic sites and increasing the phylogenetic signal} \citep{Hassanin1998611,EstoupHomoplasy}.
\hl{Also, we chose this low value to be consistent with our protocol for simulating morphological characters (see below).}
%We chose a low value of $\alpha$ to reduce the number of sites with high substitution rates, thus avoiding too much homoplasy and a decrease in phylogenetic signal. %Justify
%We selected the parameters above to generate data with no special assumption about how the characters evolved, and to reduce the computational time required if these parameters were estimated rather than defined in the tree building part of the analysis (even with the parameters defined, total computational time for the whole analysis was around 150 CPU years).
\hl{We selected this model and parameters settings to be a compromise between realism from empirical datasets} \citep[e.g.][]{kellymolecular2014} \hl{and models with few parameters} \citep[cf.][]{springermacroevolutionary2012} \hl{to reduce the computational time required if these parameters were estimated rather than defined in the tree building part of the analysis (even with the parameters defined, the total computational time for the whole analysis was around 150 CPU years).}
All the molecular information for fossil taxa was replaced by missing data ("?").

\hl{We simulated the morphological matrix using the rTraitDisc function from the R package ape v. 3.0-11} \citep{paradisape:2004}\hl{ to generate a matrix of 100 character sites for 51 taxa.}
\hl{We assigned the number of character states (either two or three) for each morphological character by sampling with a probability of 0.85 for two states characters and 0.15 for three state characters (based on an empirical review of published matrices, see Appendix A and Fig. A1 within).}
We then ran an independent discrete character simulation for each character using the ``true'' tree with the character's randomly selected number of states (two or three) and assuming an equal rate of change (i.e. evolutionary rate) from one character state to another \citep{Pagel22011994}.
This method allows us to have only two parameters for each character: the number of states and the evolutionary rate.
For each character, the evolutionary rate was sampled from a gamma distribution with $\alpha$ = 0.5.
\hl{We used low evolutionary rate parameters to be consistent with the molecular rates parameters and to avoid homoplasy in the morphological part of the matrix and create a clear phylogenetic signal} \citep{wrightbayesian2014}.
\hl{Note that} \cite{wrightbayesian2014} \hl{have shown that low morphological rates ($<0.5$) increases variance in topological error but we discarded simulations with such topological error by selecting only matrices with a ``fair'' phylogenetic signal} \citep[\hl{see Estimating phylogenies section below;}][]{zanderminimal2004}.

Finally, we combined the morphological and molecular matrices obtained from the ``true'' tree.
Hereafter we call this the ``complete'' matrix, i.e. the matrix with no missing data except for the molecular data of the fossil taxa.
%---------------------------------------------
%       2-Removing data
%---------------------------------------------

\subsection{Removing data}
% \label{Removing_data}
% We modified the ``complete'' matrix to obtain matrices with missing data by randomly replacing data with ``?" in the morphological part of the matrices according to the following parameters:
% \begin{enumerate}
% \item{$M_{L}$, the proportion of living taxa with no morphological data: 0\%, 10\%, 25\%, 50\% or 75\%.}
% This parameter illustrates the number of living taxa that are present in the molecular part of the matrix but not in the morphological part. This reflects the fact that the increased availability of molecular data means that morphological data for living species is rarely collected, and few people have the skills to identify characters needed for detailed phylogenetic analysis.
% \item{$M_{F}$, the proportion of missing data in the fossil taxa: 0\%, 10\%, 25\%, 50\% or 75\%.}
% This parameter illustrates the quality of the fossil record, a common problem due to preservation biases \citep{sansomfossilization2013}. 
% \item{$N_{C}$, the number of morphological characters for both living and fossil taxa: 100, 90, 75, 50 or 25 characters. }
% This parameter illustrates the number of available morphological characters for both living and fossil taxa. This reflects the amount of effort put into identifying the characters \citep[e.g][]{O'Leary08022013}.
% \end{enumerate}

% In practice, each parameter represents a different way of removing data from the matrix: $M_L$ removes rows from the living taxa's data; $M_F$ removes cells from the fossil taxa's data; and $N_C$ removes columns across both living and fossil taxa's data.
% Note that $M_L$ and $M_F$ differ not only because of the region of the matrix affected: for $M_L$ all the morphological data of a percentage of living taxa are removed, whereas for $M_F$ a percentage of the data are removed at random from across the whole of the morphological matrix for fossil taxa.
% We first applied the parameters $M_L$ and $M_F$ to the matrix and then applied the $N_C$ parameter.
% Therefore, when 10\% data was missing for both $M_L$ and $M_F$, 10\% data was missing in the morphological part of the matrix.
% When applying the $N_C$ parameter with the second parameter state (90 characters), however, the resulting matrix potentially had more than 10\% data missing.

% We created matrices using all parameter combinations resulting in 125 ($5^3$) matrices.
% Note that one of these combinations has no missing data so is equivalent to the ``complete'' matrix, thus we have one effectively complete matrix in our 125 ``missing-data'' matrices.
% To avoid avoid matrices containing taxa without any data (morphological or molecular), we repeated the random deletion until the matrices contained at least 5\% of data for any taxon.

% TG: this part is entirely re-written.
\hl{To explore the effect of missing data on topological recovery, we removed various amounts of the ``complete'' matrix to obtain matrices with missing morphological data.
Hereafter, we call these matrices with missing morphological data the ``missing-data'' matrices.
We removed morphological following three main data incompleteness parameters:}
\begin{enumerate}
\item{\hl{The proportion of missing living taxa ($M_{L}$).}}
\hl{This first missing-data parameter corresponds to the proportion of living taxa with no morphological data.
It represents the number of living taxa that are present in the matrix but have only molecular data available.
This reflects the fact that, because of the increasing facility of collecting molecular data, morphological data for living species is rarely collected} \citep{GuillermeCooperMissing}. % TG: see what I did here? ;)
\hl{Therefore, many living species will have only molecular data available.
In practice, we removed all the morphological data from randomly chosen living with five different proportions: 0\%, 10\%, 25\%, 50\% or 75\% of living taxa with no morphological data.}
\item{\hl{The proportion of missing data in the fossil record ($M_{F}$).}}
\hl{This missing data parameter represents the completeness of the fossil record.
In fact, due to preservation biases, missing data for fossil taxa are common }\citep{sansomfossilization2013}. 
\hl{In practice, we randomly removed a proportion of data among the fossil taxa with five different proportions: 0\%, 10\%, 25\%, 50\% or 75\% of overall missing data for the fossil taxa.}
\item{\hl{The number of morphological characters for both living and fossil taxa ($N_{C}$).}}
\hl{This parameter is not a missing data parameter \textit{per se} but rather an indication of the size of the matrix.
Any morphological matrix of any size has indeterminate missing data, given that the total number of characters is undefined, but presumably large. % TG: this is word for word what the reviewer wrote. I didn't like doing that first but after seeing the Ineterface Focus review and how they copy pasted the discussion suggestion, I think it's really nice for the reviewer to see a bit of it's work directly in the paper (at least it was for me!).
Therefore, this parameter correspond to the overall number of characters available for both living and fossil taxa.
In practice, we randomly removed whole characters from the morphological matrix shrinking it to: 100, 90, 75, 50 or 25 characters.
Note that these levels are equivalent to the two other parameters (i.e. 0\%, 10\%, 25\%, 50\% or 75\% of ``missing'' morphological characters).}
\end{enumerate}

% TG: this part is nearly exactly the same as before.
In practice, each parameter represents a different way of removing data from the \hl{morphological part of} the matrix: $M_L$ removes \hl{entire} rows from the living taxa's data; $M_F$ removes cells from the fossil taxa's data; and $N_C$ removes columns across both living and fossil taxa's data.
Note that $M_L$ and $M_F$ differ not only because of the region of the matrix affected: for $M_L$ all the morphological data of a percentage of living taxa are removed, whereas for $M_F$ a percentage of the data are removed at random from across the whole of the morphological matrix for fossil taxa.

% TG: and this one is slightly modified only.
\hl{We created matrices using all parameter combinations resulting in 125 ($5^3$) ``missing-data''matrices.
Note that one of these combinations ($M_{L}$=0\%; $M_{F}$=0\% and $N_{C}$=100) has no missing data so is equivalent to the ``complete'' matrix, thus we have one effectively complete matrix in our 125 ``missing-data'' matrices.
In practice, we first removed the data following the two missing data parameters $M_L$ and $M_F$ and then removed data following the $N_C$ parameters.
To avoid avoid matrices containing taxa without any data (morphological or molecular), we repeated the random deletion until the matrices contained at least 5\% of data for any taxon.
Note that the living taxa always had at least 90\% of data (the 1000 molecular characters).}


%---------------------------------------------
%       3-Building phylogenies
%---------------------------------------------

\subsection{Estimating phylogenies}
From the resulting matrices we generated two types of trees: the ``best'' tree inferred from the ``complete'' matrix and the ``missing-data'' trees inferred from the 125 matrices with various amounts of missing data.
The ``true'' tree was used to generate the ``complete'' matrix and reflects the ``true'' evolutionary history in our simulations.
The ``best'' tree, on the other hand, is the best tree we can build using state-of-the-art phylogenetic methods.
In real world situations, the ``true'' tree is never available to us because we cannot know the true evolutionary history of a clade \citep[except in very rare circumstances, e.g.][]{rozen2005}.
\hl{In practice, the difference between the ``true'' and the ``best'' tree represents the effect of our parameters and R packages choices (i.e. diversitree, phyclust and ape) as well as the phylogenetic methods used (see appendix B).}
\hl{The aim of this study, however, is to look at the effect the missing data parameters described above on topological recovery.}
Therefore, here we focus on comparing the trees inferred from the matrices with missing data to the ``best'' tree, rather than the ``true'' tree, as the ``best'' tree is generally what we have to work with.


\subsubsection{Maximum Likelihood}
The ``best'' tree and the ``missing-data'' trees were inferred using RAxML v. 8.0.20 \citep{Stamatakis21012014}. For the molecular data, we used the GTR + $\Gamma_4$ model (\citealp{tavare1986}; default GTRGAMMA in RAxML v. 8.0.20; \citealp{Stamatakis21012014}).
For the morphological data, we used the M\textit{kv} model \citep{lewisa2001} assuming an equal state frequency and a unique overall substitution rate ($\mu$) following a gamma distribution of the rate variation with four distinct categories (M\textit{kv} + $\Gamma_4$; -K MK option in RAxML v. 8.0.20; \citealp{Stamatakis21012014}).
We used RAxML because it automatically corrects for acquisition bias \citep{lewisa2001}. It is also heavily used in the literature for Maximum Likelihood tree inference \citep[e.g.][]{rouresite-specific2011,Bogdanowicz2012,springermacroevolutionary2012,O'Leary08022013,kellymolecular2014} and is one of the fastest methods available \citep{Stamatakis01102008}. 

To measure the support for each branch in our simulated phylogenies we first ran a fast bootstrap analysis (Lazy Sub-tree Rearrangement) with 500 replicates on the ``complete'' matrix.
We removed all the simulations with a median bootstrap support lower than 50 as a proxy for weak phylogenetic signal \citep{zanderminimal2004}.
We repeated this selection until we obtained 50 sets of simulations (i.e. 50 ``complete'' and 50 x 125 ``missing-data'' matrices) with a relatively strong phylogenetic signal (median bootstrap $>$ 50).
This step was implemented to make sure that the differences we observed in topologies (see below) were due to the amount of missing data for each parameter ($M_L$, $M_F$ and $N_C$) and not simply to low branch support that is likely to lead to different topologies.
On these selected simulations, we used the fast bootstrap algorithm and performed 1000 bootstraps for each tree inference to assess topological support \citep{pattengale2010many}.
Using these parameters took \texttildelow 8 CPU years to build 50 sets of 125 bootstrapped Maximum Likelihood trees (2.30GHz clock speed nodes). We performed this procedure to increase the resolution of our resulting trees. 

\subsubsection{Bayesian Inference}
The ``best'' tree and the ``missing-data'' trees were inferred using MrBayes v. 3.2.1 \citep{Ronquist2012mrbayes}.
We partitioned the data to treat the molecular part as a non-codon DNA partition and the morphological part as a multi-state morphological partition.
The molecular evolutionary history was inferred using the HKY model with a transition/transversion ratio of two \citep{douadycomparison2003} and a gamma distribution for the rate variation with four distinct categories (HKY + $\Gamma_4$).
For the morphological data, we used the M\textit{kv} model \citep{lewisa2001}, with equal state frequency and a unique overall substitution rate ($\mu$) with four distinct rates categories (M\textit{kv} + $\Gamma_4$).
Note that MrBayes automatically corrects for acquisition bias in the morphological data partition \citep{Nylander01022004,Ronquist2012mrbayes}.
We chose these models to be consistent with the parameters used to generate the ``complete'' matrix.

Each Bayesian tree was estimated using two runs of four chains each for a maximum of 5$\times$$10^7$ generations.
For each estimation, we used the ``true'' tree's topology as a starting tree (with a starting value for each branch length of one).
\hl{We used a fixed starting tree rather than a random starting tree} \citep[\hl{default MrBayes;}][]{Ronquist2012mrbayes} \hl{to speed up our Bayesian inferences.
Note that a starting is not a Bayesian prior on topology \textit{per se} and did not significantly affected topology (see Appendix A, section ``Effect of the starting tree on Bayesian inference'').}
We also used two priors on the molecular part of the matrix: an exponential prior on the shape of the gamma distribution of $\alpha$ = 0.5, and a transition/transversion ratio prior of two sampled from a strong beta distribution ($\beta$(80,40)); and one prior on the morphological part of the matrix (exponential prior on the shape of the gamma distribution of $\alpha$ = 0.5).
We used these priors to speed up the Bayesian estimation process.
These priors biased the way the Bayesian process calculated branch lengths by giving non-random starting points and boundaries for parameter estimation however, here we are focusing on the effect of missing data on tree topology and not branch lengths.
Even using these priors, it took $~$ 140 CPU years to build 50 sets of 125 Bayesian trees (2.30GHz clock speed nodes).
\hl{The detailed MrBayes parameters are available in Appendix A along with details on the $\alpha$ parameter estimation.}

We used the average standard deviation of split frequencies (ASDS) as a proxy to estimate the convergence of the chains and used a stop rule when the ASDS went below 0.01 \citep{Ronquist2012mrbayes}.
We also checked the effective sample size (ESS) on a random sub-sample of runs in each simulation to ensure that ESS $>>$ 200 \citep{drummond2006ess}.
Finally we built a strict majority rule Bayesian consensus tree from the combined chains, excluding the 25\% first iterations as burn-in \citep{Ronquist2012mrbayes}.

%---------------------------------------------
%       4. Comparing Topologies
%---------------------------------------------

\subsection{Comparing topologies}
We compared the topology of the ``missing-data'' trees to the ``best'' tree to measure the effect of the three parameters $M_{L}$, $M_{F}$ and $N_{C}$ on tree topology.
We used the Robinson-Foulds distance \citep{RF1981} to assess the amount of conserved clade positions and the Triplets distance \citep{dobson1975triplets} to assess the amount of wildcard taxa \citep[i.e. taxa that frequently change position in different trees][]{kearneyfragmentary2002}. We used these two metrics because they illustrate two different aspects of tree topology (see Discussion) but also because their performance in measuring differences in topology is well described \citep{Kuhner04112014} and well implemented \citep{Bogdanowicz2012}.
We normalised both metrics using methods described in \citet{Bogdanowicz2012} to generalize our results for any \textit{n} number of taxa.
These metrics are described in detail below.

\subsubsection{Robinson-Foulds distance}
The Robinson-Foulds distance \citep{RF1981}, or ``path difference", measures the difference between the number of clades and \hl{twice the number} of shared clades across two trees.
The metric reflects the distance between the distributions of tips among clades in the two trees \citep{RF1981} (see Appendix B for calculation details).
This metric is bounded between zero, when the two trees are identical, and $2(n-2)$ (for two trees with $n$ taxa) when there is no shared clade in the two trees.
This metric is sensitive to minor changes in clade conservation: if the trees are composed of two clades of three taxa (\textit{(((a,b),c),((d,e),f))}), the swapping of any two taxa will lead to a maximal score of the Robinson-Foulds distance indicating poor tree similarity.
We normalised this metric following Bogdanowicz's Normalised Tree Similarity (NTS) method \citep{Bogdanowicz2012}.
This method scales any tree comparison metric using the mean distance between 1000 random trees (see Appendix B for the calculation details).
This method is a generalisation of the topological accuracy method \citep{Price2010} allowing to compare topological differences between any tree with any tree comparison metric.
In practice when the Normalised Robinson-Foulds metric between two trees is equal to one, the trees are identical; if the metric is equal to zero, the trees are no more different than expected by chance; finally if the metric is less than zero, the trees are more different than expected by chance.
Note that once rescaled, the Normalised Robinson-Foulds metric is a measure of similarity, rather than of distance like the original Robinson-Foulds metric. 

\subsubsection{Triplets distance}
The Triplets distance \citep{dobson1975triplets} measures the number of sub-trees made up of three taxa that differ between two trees \citep{critchlowthe1996} (see Appendix B for calculation details).
This metric measures the position of each taxon and clade in relation to its closest neighbours.
It is bounded between zero when the two trees are identical and $\binom{n}{3}$ (for two trees with $n$ taxa) when there is no shared taxa/clade position in the two trees.
Therefore this metric is sensitive to the conservation of wildcard taxa.
We normalised this metric in the same way as for the Robinson-Foulds distance resulting in the Normalised Triplets metric.

\subsubsection{Paired tree comparisons}
\label{tree_comparisons}
For the Maximum Likelihood and Bayesian consensus trees we performed pairwise comparisons between the ``best'' tree and each ``missing-data'' tree using both the Normalised Robinson-Foulds and Normalised Triplets metrics with the TreeCmp java script \citep{Bogdanowicz2012} resulting in 125 Normalised Robinson-Foulds metrics and 125 Normalised Triplets metric for each tree inference method.
Also, to take into account the uncertainty of tree inference, we extracted 1000 random bootstrapped trees from the Maximum Likelihood analysis and 1000 trees from the posterior tree distribution of the Bayesian analysis for the ``best'' trees, and then did the same for the 125 ``missing data" trees (resulting in 1000 ``best'' trees and 125$\times$1000 ``missing data" trees). 
For a given set of 1000 ``missing data" trees and the 1000 ``best'' trees, we sampled one ``missing data" tree and one ``best'' tree at random and compared them using both the Normalised Robinson-Foulds and Normalised Triplets metrics as described above.
We repeated this 1000 times for each set of ``missing data" trees resulting in 125$\times$1000 values for each metric.
We repeated all the paired tree comparisons described above for each of the 50 simulation runs.
We then calculated the mode and the 50\% and 95\% confidence intervals from the resulting distribution using the hdrcde R package v. 3.1 \citep{hdrcde}.

\subsection{Testing the effects of the missing data parameters on topological recovery}
Finally, we tested the effects of our missing data parameters ($M_{L}$, $M_{F}$, $N_{C}$ and their interactions) on our ability to recover the ``best'' tree topology in a Total Evidence framework.
We also assessed the effect of our missing data parameters jointly with the effects of different tree inference and uncertainty methods (i.e. Maximum Likelihood, Bayesian consensus, Maximum Likelihood bootstrap trees and Bayesian posterior tree distribution).

We measured similarities among the distributions of the different metrics scores (Normalised Robinson-Foulds and Normalised Triplets metric) using the Bhattacharyya Coefficient \citep{Bhattacharyya}.
The Bhattacharyya Coefficient is the probability of overlap between two distributions \citep{Bhattacharyya} (see Appendix B for calculation details).
Note that this is comparable to performing a two-sided t-test, but we use the Bhattacharyya Coefficient here because we are comparing whole distributions not just their means.
\hl{When the Bhattacharyya Coefficient between two distributions is $<$0.05, the distributions are significantly different.
When this coefficient is $>$0.95 both distributions are significantly similar.
Values in between these two threshold just show the probability of overlap between the distributions but are not conclusive to assess the similarity or differences between the distributions.}
\hl{To assess the effect of our missing data parameters, we calculated the Bhattacharyya Coefficient between the distributions of the different metrics scores (Normalised Robinson-Foulds and Normalised Triplets metric) for each pairwise combination of missing data parameters ($M_{L}$, $M_{F}$, $N_{C}$) and parameter states (0\%, 10\%, 25\%, 50\%, 75\% and 100, 90, 75, 50, 25 characters), i.e. $M_{L}$ = 0\%, $M_{F}$ = 0\%, $N_{C}$ = 100; $M_{L}$ = 10\%, $M_{F}$ = 0\%, $N_{C}$ = 100 etc. (see Figure 2 for more details).}
This resulted in 7875 pairwise comparisons (a triangular matrix with $3^5$$\times$$3^5$ cells).
We performed this procedure separately for each tree inference and uncertainty method.
When two combinations of missing data parameters have a similar ability to recover the ``best'' tree topology the Bhattacharyya Coefficient will be close to one.
Conversely, if the two combinations of missing data parameters differ, the Bhattacharyya Coefficient will be close to zero.
\hl{Because of the difficulties to represent so many pairwise comparisons in a meaningful way, we summarized as a heat map of Bhattacharyya Coefficients (see Figure 6).
In this type of figures, parameters that have similar or different effects on recovering the ``best'' topology (rather good or bad effects) will be denoted by similar colour patches in the heat map representation of these comparisons (see Figure 6).}

%\begin{figure}[h]
%\caption{{\bf Bhattacharyya Coefficient calculation outline 1.}
% A, B and C are distributions of tree similarity metrics (Normalised Robinson-Foulds or Normalised Triplets metrics) for any combination of missing data parameters (e.g. $M_{L}$ = 10\%, $M_{F}$ = 50\%, $N_{C}$ = 75\%). The Bhattacharyya Coefficient (BC) is the overlap of the distribution of tree similarity metrics between two combinations of missing data parameters, for example, BC(A,B) is the probability of overlap between the distributions A and B. Note that this is similar to performing a two-sided t-test, but we use the Bhattacharyya Coefficient here because we are comparing distributions not means.}
%\label{Fig_Bhattacharyya_Coefficients1}
%\end{figure}

To assess the effect of the different tree inference and uncertainty methods (i.e. Maximum Likelihood, Bayesian consensus, Maximum Likelihood bootstrap trees and Bayesian posterior tree distribution) on our ability to recover the ``best'' tree topology, we calculated the Bhattacharyya Coefficient between the distributions of the different metrics scores (Normalised Robinson-Foulds and Normalised Triplets metric) for each pairwise combination of tree inference and uncertainty methods, i.e. Maximum Likelihood \textit{versus} Bayesian consensus; Maximum Likelihood \textit{versus} Maximum Likelihood bootstrap trees etc. (see Figure 3 for more details).
Note that this procedure pools results from across all missing data parameter combinations so it results in just six pairwise comparisons.
When two tree inference or uncertainty methods have a similar ability to recover the ``best'' tree topology the Bhattacharyya Coefficient will be close to one.
Conversely, if the two tree inference or uncertainty methods differ, the Bhattacharyya Coefficient will be close to zero.

%\begin{figure}[h]
%\caption{{\bf Bhattacharyya Coefficient calculation outline 2.}
 %A and B are distributions of tree similarity metrics (Normalised Robinson-Foulds or Normalised Triplets metrics) for any combination of missing data parameters (e.g. $M_{L}$ = 10\%, $M_{F}$ = 50\%, $N_{C}$ = 75\%). \textbf{(x)} and \textbf{(y)} are two different tree inference methods (e.g. Maximum Likelihood or Bayesian). The Bhattacharyya Coefficient (BC) is the overlap of the distribution of tree similarity metrics between two methods for the same combination of missing data parameters, for example, BC($A_{x}$,$A_{y}$) is the probability of overlap of the distribution A for methods $x$ and $y$. Note that this is similar to performing a two-sided t-test, but we use the Bhattacharyya Coefficient here because we are comparing distributions not means.}
%\label{Fig_Bhattacharyya_Coefficients2} 
%\end{figure}

%---------------------------------------------
%
%       RESULTS
%
%---------------------------------------------

\section{Results}
As the amount of missing data in the morphological part of the Total Evidence matrix increases, our ability to recover the ``best'' tree topology decreases, regardless of the missing data parameter ($M_{L}$, $M_{F}$ or $N_{C}$), the tree inference method (Maximum Likelihood or Bayesian) or the tree comparison metric used (Normalised Robinson-Foulds or Normalised Triplets metric).
Nonetheless, the different missing data parameters and tree inference methods do not affect the topology in the same way (Figure 4 and Figure 5).

%\begin{figure}[]
%\caption{{\bf The effects of increasing missing data on topological recovery using Maximum Likelihood trees (black), Bayesian consensus trees (blue), Maximum Likelihood bootstrap trees (orange) and Bayesian posterior tree distributions (blue).}
 %The percentage of missing data for each parameter ($M_{L}$, $M_{F}$ and $N_{C}$) is shown on the x axis. Topological recovery was measured using two different tree comparison metrics: Normalised Robinson-Foulds metric (upper row) and Normalised Triplets metric (lower row). The graph shows the modal value (points), and the 50\% (thick solid lines) and 95\% (thin dashed lines) confidence intervals of the distributions of the tree comparison metric for each missing data parameter and tree inference method.}
%\label{Fig_Results-permeth_perparam} 
%\end{figure}

%\begin{figure}[]
%\caption{{\bf The effects of increasing missing data on topological recovery using Maximum Likelihood trees (black) and Bayesian consensus trees (grey).}
 %The x axis shows the percentage of missing data from 0\% (white) to 75\% (black) for the three parameters: $M_{L}$ (upper line), $M_{F}$ (middle line) and $N_{C}$ (lower line). Topological recovery was measured using two different tree comparison metrics: Normalised Robinson-Foulds metric (upper row) and Normalised Triplets metric (lower row). The graph shows the modal value (points), and the 50\% (thick solid lines) and 95\% (thin dashed lines) confidence intervals of the distributions of the tree comparison metric for each missing data parameter and tree inference method.} 
%\label{Fig_Results-global_perparam}
%\end{figure}

\subsection{Individual effects of missing data parameters}
As the amount of missing data increases across all three parameters, our ability to recover the ``best'' tree topology decreases (Figure 4).
The Normalised Robinson-Foulds metric is always lower for the Maximum Likelihood trees than for the Bayesian consensus trees (median Bhattacharrya Coefficient = 0.69, 0.48 and 0.66 for $M_{L}$, $M_{F}$ and $N_{C}$ respectively; Figure 4; Tables C5, C6 and C7 in Appendix C). 
The Normalised Triplets metric, however, is similar between the Maximum Likelihood trees and the Bayesian consensus trees for all the parameters ($M_{L}$, $M_{F}$ and $N_{C}$) (median Bhattacharrya Coefficient = 0.84, 0.75 and 0.80 for $M_{L}$, $M_{F}$ and $N_{C}$ respectively; Figure 4; Tables C5, C6 and C7 in Appendix C).

\subsection{Combined effect of missing data parameters}
As expected, our ability to recover the ``best'' tree topology is worst when each parameter contains the maximum amount of missing data (i.e. $M_{L}$ = 75\%, $M_{F}$ = 75\% and $N_{C}$ = 75\%), and best when there is no missing data (i.e. $M_{L}$ = 0\%, $M_{F}$ = 0\%, $N_{C}$ = 0\%; Figure 5; Tables C2, C3 and C4 in Appendix C).
Figure 6 shows the similarity of distributions of tree metrics in a triangular matrix with the values of each pairwise Bhattacharyya Coefficient coloured according to their values (orange when the distributions overlap completely, Bhattacharyya Coefficient = 1, and blue when they do not, Bhattacharyya Coefficient = 0). 

%\begin{figure}[]
%\caption{{\bf The effects of missing data on topological recovery using Bayesian consensus trees.}
% Both axes show the percentage of missing data from 0\% (white) to 75\% (black) for the three parameters: $M_{L}$ (upper line), $M_{F}$ (middle line) and $N_{C}$ (lower line). Topological recovery is represented by the probability of (A) Normalised Robinson-Foulds metric and (B) Normalised Triplets metric distributions overlapping with the ``best'' tree distribution, calculated using the Bhattacharyya Coefficient. The Bhattacharyya Coefficient values are indicated using a color gradient ranging from low probability of overlap in blue, to a high probability of overlap in orange.
%}
%\label{Fig_Results-paircomp_within}
%\end{figure}

\hl{Using both Normalised Robinson-Foulds and Normalised Triplets metrics from the Bayesian consensus trees, the parameter combination with no missing data (i.e. $M_{L}$ = 0\%, $M_{F}$ = 0\%, $N_{C}$ = 100) is always the most dissimilar to all the other parameter combinations (thin deep blue line at the base of Figure 6).
The Normalised Robinson-Foulds metric (median Bhattacharrya coefficient = 0.79; blue regions in Figure 6A), however, displays more dissimilarities than the Normalised Triplets metric (median Bhattacharrya coefficient = 0.81; blue regions in Figure 6B).
The orange upper triangle in Figure 6A shows a high probability of overlap of the Normalised Robinson-Foulds metric for the trees with the $M_{L}$ parameter $\geq$ 50\% (Figure 6A).
There is no additional effect (i.e. no dissimilarities) of $M_{F}$ and $N_{C}$, regardless of the amount of missing data in these parameters (Figure 6A).
Additionally, when using the Normalised Robinson-Foulds metric, once $M_{L}$ $\geq$ 50\%, there is no additional effect of $M_{F}$ and $N_{C}$, regardless of the amount of missing data in these parameters (Figure 6A).
Likewise, once $N_{C}$ $<$ 50, there is no additional effect of $M_{L}$ and $M_{F}$ as denoted by the high probability of Normalised Robinson-Foulds metric overlap (horizontal orange stripes between the blue regions Figure 6A).
This can be interpreted as, in figure 5 for the Normalised Robinson-Foulds metric, the overlap between the distributions once $M_L$=50\%.}

For all combinations of missing data parameters and tree comparison metrics, the Maximum Likelihood bootstrap trees and the Bayesian posterior tree distributions perform very similarly (median Bhattacharrya Coefficient = 0.85 and 0.98, using Normalised Robinson-Foulds metric or Normalised Triplets metric respectively; Table ~\ref{Tab_Results-Difference_methods}).
These two methods, however, perform worse than the Bayesian consensus trees using Normalised Robinson-Foulds metric (median Bhattacharrya Coefficient = 0 and 0.01, for the Maximum Likelihood bootstrap trees and the Bayesian posterior tree distribution respectively; Table ~\ref{Tab_Results-Difference_methods}; Figure 4 and Figure C2 in Appendix C).

%---------------------------------------------
%
%       Discussion
%
%---------------------------------------------


\section{Discussion}

Our results show that the ability to recover the ``best'' tree topology in a Total Evidence framework decreases as the amount of missing data increases, regardless of how data were removed or the method of tree inference used.
These factors, however, affected topological recovery in different ways and to different extents.
Decreasing the amount of living taxa with morphological data ($M_{L}$) and the overall number of morphological characters in the matrix ($N_{C}$) had worst effects on topological recovery (Figure 6).
Additionally, using Bayesian consensus trees recovered the ``best'' tree topology more consistently than using Maximum Likelihood trees or Bayesian posterior tree distributions (Figure 5, Figure 6, Table ~\ref{Tab_Results-Difference_methods}).
As seen in previous studies, our results show that the amount of missing data is not a problem \textit{per se} for Total Evidence methods, as long as enough living and fossil taxa in the matrix have data for overlapping morphological characters \citep[e.g.][]{kearneyfragmentary2002,wiensmissing2003,rouresite-specific2011,pattinsonphylogeny2014}.

\subsection{Individual effects of missing data parameters}
\subsubsection{Missing data for living taxa ($M_{L}$)}
When the number of living taxa with morphological data ($M_{L}$) decreases, entire rows of data are being removed from the living taxa part of the matrix.
Because living taxa still have molecular characters available for phylogenetic inference (see Methods), even if they have no morphological data, the relationships among them will always be fairly well-resolved (depending on the phylogenetic signal from the molecular part of the matrix).
This missing data parameter, however, has a huge influence on the placement of fossil taxa because a decrease in the $M_{L}$ parameter reduces the amount of overlapping data among the living and fossil taxa, meaning there is no part of the living taxa tree that the fossils can branch off.

\subsubsection{Missing data for fossil taxa ($M_{F}$)}
When the overall proportion of data for the fossil taxa ($M_{F}$) decreases, this also reduces the probability of morphological characters for fossil taxa overlapping with the ones for living taxa.
This can lead to difficulties for the placement of certain taxa in the tree.
It is important, however, to note that even though the number of displaced wildcard taxa increases (i.e. decrease of Normalised Triplets metric) with increasing missing data in this parameter, clade conservation (i.e. Normalised Robinson-Foulds metric) is still relatively good (mode = 0.72) when the proportion of missing data is high ($M_{F}$ = 75\%).

The effect of the missing data in the fossil record ($M_{F}$) is less than the effect of the $M_{L}$ parameter on clade conservation (Normalised Robinson-Foulds metric) but greater on the displacement of wildcard taxa (Normalised Triplets metric; Figure 4 and Figure 5).
This is related to the fact that the Bayesian consensus tree is built using a majority consensus rule.
When the fossil taxa have less data (e.g. $M_{F}$ = 75\%) they will tend to branch with any taxon in the clade that shares most characters with the fossils.
Therefore a majority consensus position is unlikely to exist (i.e. every branching position is represented in $<$ 50\% of the trees in the Bayesian posterior distribution) and the fossil taxa will form a polytomy at the base of the clade.
\hl{In this case, the Normalised Robinson-Foulds metric will decrease when the fossil is present near the tips but affects the clade conservation less when fossils are near the root.}
Conversely, because a fossil in a high taxonomic level clade has many chances to branch on different nodes within the clade, it will be more likely to act as a wildcard taxon and decrease the Normalised Triplets metric.
Therefore, the $M_{F}$ parameter is likely to affect the Normalised Robinson-Foulds metric less than the Normalised Triplets metric for the Bayesian consensus trees.
Conversely, the same scenario in a Maximum Likelihood framework will lead to a dichotomous branching of the fossils but with low bootstrap support ($<$ 50).
In other words, the Bayesian consensus tree allows a fossil taxon with few data to be placed with a higher confidence at a lower taxonomic level than the Maximum Likelihood tree, where the fossil will be placed with lower confidence at a higher taxonomic level.
We argue that using the Bayesian consensus tree topology is preferable because it is more conservative \citep[e.g.][]{pattinsonphylogeny2014}.

\subsubsection{Number of morphological characters ($N_{C}$)}
Reducing the overall number of morphological characters reduces the probability of their overlap among the taxa in the matrix, and therefore decreases our ability to recover the ``best'' tree topology.
We expected the decrease in this parameter to have an effect twice as large as that for the $M_{L}$ and $M_{F}$ parameters, because removing 10\% of the data for the fossil or living taxa only removes 5\% of data from the whole matrix (because this parameter affects only half of the taxa present in the matrix).
Conversely, removing 10\% of morphological characters (\hl{i.e. $N_{C}$ = 90}) genuinely removes 10\% of data in the matrix.
Nonetheless, the effect of removing characters on the ability to recover the ``best'' tree topology is of the same order of magnitude as for the other two parameters (Figure 4).
We suspect this again reflects the importance of overlapping characters, as opposed to the number of characters \textit{per se}.

Additionally, the number of morphological characters determines the size the of matrix.
This can affect our ability to recover the ``best'' tree topology through: (1) the incongruence of phylogenetic signal among morphological and molecular data; and/or (2) homoplasy.
The incongruence of phylogenetic signal between morphological and molecular data has previously been demonstrated to be more important in small morphological matrices \hl{(}\citealt{bremer1992phylogeny,patterson1993congruence}\hl{; see }\citealt{masters2002lack}\hl{ for a empirical example)}.
\hl{The sizes of our data matrices were constrained by the performance of our protocol}: to reduce the computational time of our analysis to a reasonable level (150 CPU years), we ran our simulations on modestly-sized matrices of 1000 molecular characters and 100 morphological characters.
Therefore, part of the decrease of the Normalised Robinson-Foulds metric and the Normalised Triplets metric in our simulations could be due to conflicting phylogenetic signal among morphological and molecular data in our matrices (Figure 4 and Figure 5).
Although these matrices are an order of magnitude smaller than some published matrices \citep[e.g.][]{springermacroevolutionary2012,nithe2013}, they are still within the size range of more modestly-sized empirical matrices \citep[e.g.][]{kellymolecular2014, sallam2011craniodental}.
Therefore, our simulations reflect realistic parameters.
Nonetheless, the use of probabilistic methods (i.e. Maximum Likelihood or Bayesian) and the M\textit{kv} model \citep{lewisa2001} has been previously demonstrated to partially resolve this issue \citep{wrightbayesian2014}.

\subsection{Combined effect of missing data parameters}
As expected, when combining the missing data parameters, our ability to recover the ``best'' tree topology is affected in the same way as for the parameters individually: the Normalised Robinson-Foulds metric and the Normalised Triplets metric are higher when all the missing data parameters have few missing data (i.e. $M_{L}$ = 0\%, $M_{F}$ = 0\%, $N_{C}$ = 100) and lower when \hl{they have a larger proportion of missing data} (i.e. $M_{L}$ = 75\%, $M_{F}$ = 75\% and $N_{C}$ = 25; Figure 5).
It is important, however, to notice that the effect of each parameter is not additive.
\hl{Surprisingly, the number of missing living taxa with morphological data ($M_{L}$) and the overall number of missing morphological characters ($N_{C}$), have a bigger effect than the amount of missing data for the fossil taxa ($M_{F}$).
For any additional missing living taxa with morphological data ($M_L$) beyond 50\%, there is no difference between trees with any combinations of the other parameters ($M_F$ and $N_C$; Figure 6).
In other words, when the number of missing living taxa reaches 50\%, the amount of missing data in the fossil record ($M_F$) or the number of characters ($N_C$) doesn't matter and and increase/decrease in both does not affect topology.
A similar effect can be observed when the $N_C$ parameter reaches 50 characters (Figure 6).
This has important practical implications, especially on the best strategy to improve topology by collecting morphological characters (see below).}

\subsection{Effects of tree inference methods}
Variation in our ability to recover the ``best'' tree topology depends heavily on the tree inference method (Figure 4 and Figure 5).
For morphological data, previous studies have shown some superiority of probabilistic tree inference methods with simple evolutionary models such as the M\textit{kv} model \citep{lewisa2001} over \hl{parsimony} methods (\citealp{wrightbayesian2014}; but see \citealp{spencerefficacy2013}).
This is, however, the first study, to our knowledge, to compare the performance of the M\textit{kv} model \citep{lewisa2001} for recovering the ``best'' tree topology using Maximum Likelihood and Bayesian methods in a Total Evidence framework.
\hl{Our results show that the topology of the Bayesian consensus tree is always closer to the ``best'' tree topology than the ``best'' Maximum Likelihood tree (Figure 5).
Note that the methodological choice of using the ``true'' tree as a starting tree for the Bayesian Inference (see Methods) had no significant effect on topological recovery (see Appendix A, section ``Effect of the starting tree on Bayesian inference'') for details on the analysis of the effect of the starting tree in Bayesian inference).}
As described above, this is because the Bayesian consensus tree allows a fossil taxon with few data to be placed with a higher confidence at a lower taxonomic level than the Maximum Likelihood tree.
This may also be because the ``best'' Bayesian consensus trees are not completely resolved, thus will always be more similar to the ``missing data" trees than a completely resolved tree like the ``best'' Maximum Likelihood tree.
Nonetheless, we minimized the probability of unresolved ``best'' trees in our Bayesian analyses by only using datasets with strong phylogenetic signal (see Methods).

\hl{The Bayesian consensus trees, however, performs badly for the Normalised Triplets metric: some parameters combinations, especially when the $M_F$ parameter reaches 75\% missing data, leads to negative values (figure 5).
Normalised Triplets metric values below 0 means that the placement of some taxa is worth than expected by just randomly placing this taxa in the tree.
This can be interpreted by the absence of comparable triplets between some of the ``missing data" trees and ``best'' trees.
In fact, even if clades are conserved (figure 5), the resolutions within them can be poor to non-existent when a lot of data is missing (i.e. 75\%).
In such cases, the fossil taxa are likely to be placed in any of the clades that share the most characters.
These results are in agreement with previous studies that have showed that missing data can be deleterious for recovering ``correct'' topologies, especially for small matrices of 100 characters }\citep{wiensmissing2003}.
\hl{It is important to note, however, that this effect can be reduced by just increasing the number of characters }\citep{wiensmissing2003}.


It is also worth noting that across all our analyses, the topologies of the Maximum Likelihood bootstrap trees and the Bayesian posterior trees distribution were always further from the ``best'' tree topology than Maximum Likelihood and Bayesian consensus trees.
This was true even when no morphological data was missing ($M_{L}$ = 0\%; $M_{F}$ = 0\%, $N_{C}$ = 100; Figure 4).
This reflects the fact that it is difficult to compare two distributions of trees, and each comparison between a set of ``missing data" trees and a set of the ``best'' trees involved 1000 random pairwise comparisons rather than just one.
\hl{Additionally, the Bayesian posterior trees performed way worth than the Bayesian consensus tree (figure 4, table 1 and appendix C figure C5 and tables C5, C6 and C7).
This can be due to the fact that the Bayesian posterior trees are always resolved and thus more likely to contain wrongly resolved nodes (i.e. decreasing the Normalised Robinson-Foulds metric).
Conversely, the Bayesian consensus tree might not resolve nodes that are badly supported and thus more likely to contain only correctly resolved nodes (i.e. increasing the Normalised Robinson-Foulds metric).}


\subsection{Practical implications}
Our missing data parameters illustrate different sources of missing data in empirical matrices as follows: ($M_{L}$) the paucity of coded morphological characters for living taxa; ($M_{F}$) the missing data for fossils (or parts of fossils) that have not been preserved in the fossil record; and ($N_{C}$) characters that have not been coded across living and fossil species, perhaps due to difficulties in coding or poor preservation of the feature in collections.
Filling these gaps in empirical Total Evidence matrices should lead to a substantial increase in our ability to recover the ``best'' tree topology.
We can increase the number of living taxa with coded morphological characters by increasing research efforts in this area, and encouraging use of our vast natural history collections.
Increasing data for fossil species is harder, since it depends on fossil preservation biases and new fossil discoveries.
Gaps in the matrix, however, can be filled with efforts in palaeontological field work that can potentially lead to future discoveries of exceptionally preserved fossils \citep[e.g.][]{nithe2013}.
Fortunately, although this data is the most difficult to collect, it also has the least influence on whether our simulations recover the ``best'' tree topology (Figure 6).
Finally, although increasing the number of coded characters is relatively straightforward, the amount of time it takes to build a morphological matrix increases directly with the number of characters involved.
One solution to this problem may be to engage with collaborative data collection projects through web portals such as \textit{MorphoBank} \citep{morphobank}, so that no one individual collects all the data.

\hl{Another practical implication of our results regards the tree inference methods.
Because the Bayesian consensus trees consistently recovered topologies closer to the ``best'' tree topology than the Maximum Likelihood trees, we advice using the Bayesian consensus trees as a topological constraint for tree inferences using the Total-Evidence method such as tip-dating} \citep[\hl{e.g.}][\hl{; although it is possible that including dating information during tree inference could also improve the accuracy of the Bayesian posterior tree distribution}]{ronquista2012,Wood01032013,BEASTmaster}.
\hl{Additionally, using the Bayesian consensus tree rather than the Maximum Likelihood can reduce the amount of ``false positive'' topologies.
As shown in figure 5 and discussed in the section above (Effects of tree inference methods), the Bayesian consensus tree is more likely to not resolve nodes weakly supported due to missing data than the Maximum Likelihood tree that is more likely to wrongly resolve such nodes (i.e. creating a ``false positive'' node).
Note, however, that we do not suggest discarding the Bayesian posterior tree distribution once one chose to use the Bayesian consensus tree topology even though they performed poorly in recovering the ``best'' tree topology in our simulations (this can probably be imputed to the difficulties comparing distributions of trees; see above).
This can be particularly important because these trees will be invaluable for phylogenetic comparative analyses.
For example a sub-sample of posterior trees distributions can be used to asses macroecological questions while better taking into account topological uncertainty (e.g. }\citealt{FritzTree}\hl{ and }\citealt{jetzthe2012}\hl{ trees used in }\citealt{healy2014}\hl{).}

\section{Conclusions}
\hl{Previous studies have explored the effect of missing morphological data when combining living and fossil species in phylogenies but where limited by their empirical approach or their aspects of missing data }\citep{Wiens01102005,pattinsonphylogeny2014}.
\hl{This study proposes a more in depth approach where missing data is generated on simulated data and along three clear missing-data parameters ($M_{L}$, $M_{F}$ or $N_{C}$).
This allowed use to confirm previous results that missing data can be specially deleterious in small sized matrices }\citep{wiensmissing2003}\hl{ but also relieved a yet unexplored aspect: the crucial importance of coding morphological data for living species.}
\hl{In fact, missing data in Total Evidence matrices is not a problem for recovering the ``best'' tree topology as long as enough living and fossil taxa in the matrix have data for overlapping morphological characters.
When missing data increases in any of our missing data parameters ($M_{L}$, $M_{F}$ or $N_{C}$), it reduces support for the placement of fossil taxa and increases the displacement of wildcard taxa.
Therefore we advise that one should focus on having morphological characters coded for a large amount of living taxa present in the matrix (i.e. 50\%) for accurately combining both living and fossil species in phylogenies.
Doing so grants a good overlap of morphological characters between living and fossil taxa, allowing the fossil taxa to be positioned relatively to the living taxa based on their actual shared derived characters rather than on simply available data.}

\hl{Additionally, the topology of the Bayesian consensus trees, regardless the amount of missing data, were always closer to the ``best'' tree topology than the Maximum Likelihood trees.
This has also been observed in empirical data} \cite[\hl{e.g.}][]{Arcila2015131}\hl{ were Maximum Likelihood trees inferred from a Total Evidence matrix were less supported than the Bayesian consensus tree.
This might have an important impact on estimating topologies in the Total Evidence framework since previous studies had to rely either on molecular scaffolds }\citep[\hl{e.g.}][]{slaterphylogenetic2013}\hl{, taxonomic constraints }\citep[\hl{e.g.}][]{slaterphylogenetic2013,beckancient2014}\hl{ or even by fixing the topology }\cite[\hl{e.g.}][]{ronquista2012}.
\hl{Therefore, we suggest extracting such topological frames from the Bayesian consensus tree if needed.}

\hl{To conclude, the results of our analyses are encouraging and show that it is possible to accurately combine both neontological and palaeontological data in the same phylogeny as long as both types of data sufficiently overlap.
Hopefully, using these approaches will greatly improve our understanding of macroevolutionary patterns and processes.}


\section{Acknowledgments}
Thanks to Gavin Thomas, Fr\'{e}d\'{e}ric Delsuc, Emmanuel Douzery, Trevor Hodkinson, Andrew Jackson, Nick Matzke, and April Wright for useful comments on our simulation protocol and manuscript. Thanks to Paddy Doyle, Graziano D'Innocenzo and Sean McGrath for assistance with the computer cluster. Thanks to the two anonymous reviewers for their useful and enthusiastic % TG: or "useful and encouraging"? I feel just useful does gives justice to them, especially to the work of reviewer 1...
comments. Simulations used the Lonsdale cluster maintained by the Trinity Centre for High Performance Computing and funded through grants from Science Foundation Ireland. This work was funded by a European Commission CORDIS Seventh Framework Programme (FP7) Marie Curie CIG grant (proposal number: 321696).

\nolinenumbers

\bibliographystyle{sysbio}
\bibliography{References}

\newpage
\noindent
\textbf{Figure captions}\\
\bigskip

\noindent
\textbf{Figure 1:} Protocol outline.
(1) We randomly generated a birth-death tree (the ``true'' tree) and used it to simulate a matrix with no missing data (the ``complete'' matrix).
(2) We removed data from the morphological part of the ``complete'' matrix resulting in 125 ``missing-data'' matrices.
(3) We built phylogenetic trees from each matrix using both Maximum Likelihood and Bayesian methods.
(4) We compared the ``missing-data'' trees to the ``best'' tree.
We repeated these four steps 50 times.\\
\bigskip

\noindent
\textbf{Figure 2:} Bhattacharyya Coefficient calculation outline 1. A, B and C are distributions of tree similarity metrics (Normalised Robinson-Foulds or Normalised Triplets metrics) for any combination of missing data parameters (e.g. $M_{L}$ = 10\%, $M_{F}$ = 50\%, $N_{C}$ = 25). The Bhattacharyya Coefficient (BC) is the overlap of the distribution of tree similarity metrics between two combinations of missing data parameters, for example, BC(A,B) is the probability of overlap between the distributions A and B.\\

\noindent
\textbf{Figure 3:} Bhattacharyya Coefficient calculation outline 2. A and B are distributions of tree similarity metrics (Normalised Robinson-Foulds or Normalised Triplets metrics) for any combination of missing data parameters (e.g. $M_{L}$ = 10\%, $M_{F}$ = 50\%, $N_{C}$ = 25). \textbf{(x)} and \textbf{(y)} are two different tree inference methods (e.g. Maximum Likelihood or Bayesian). The Bhattacharyya Coefficient (BC) is the overlap of the distribution of tree similarity metrics between two methods for the same combination of missing data parameters, for example, BC($A_{x}$,$A_{y}$) is the probability of overlap of the distribution A for methods $x$ and $y$.\\

\bigskip
\noindent
\textbf{Figure 4:} The effects of increasing missing data on topological recovery using \hl{Maximum Likelihood trees (black), Bayesian consensus trees (grey), Maximum Likelihood bootstrap trees (blue) and Bayesian posterior tree distributions (orange)}. The percentage of missing data for each parameter ($M_{L}$, $M_{F}$ and $N_{C}$) is shown on the x axis. Topological recovery was measured using two different tree comparison metrics: Normalised Robinson-Foulds metric (upper row) and Normalised Triplets metric (lower row). The graph shows the modal value (points), and the 50\% (thick solid lines) and 95\% (thin dashed lines) confidence intervals of the distributions of the tree comparison metric for each missing data parameter and tree inference method.


\bigskip
\noindent
\textbf{Figure 5:} The effects of increasing missing data on topological recovery using Maximum Likelihood trees (black) and Bayesian consensus trees (grey). The x axis shows the percentage of missing data from 0\% (white) to 75\% (black) for the two parameters: $M_{L}$ (upper line), $M_{F}$ (middle line) and number of characters from 100 to 25 for the parameter $N_{C}$ (lower line). Topological recovery was measured using two different tree comparison metrics: Normalised Robinson-Foulds metric (upper row) and Normalised Triplets metric (lower row). The graph shows the modal value (points), and the 50\% (thick solid lines) and 95\% (thin dashed lines) confidence intervals of the distributions of the tree comparison metric for each missing data parameter and tree inference method.

\bigskip
\noindent
\textbf{Figure 6:} \hl{The effects of missing data on topological recovery using Bayesian consensus trees. Both axes show the percentage of missing data from 0\% (white) to 75\% (black) for the two parameters: $M_{L}$ (upper line), $M_{F}$ (middle line) and number of characters from 100 to 25 for the parameter $N_{C}$ (lower line). The topological recovery is measured as (A) the Normalised Robinson-Foulds metric and (B) the Normalised Triplets metric calculated using the Bhattacharyya Coefficient. The Bhattacharyya Coefficient values are indicated using a color gradient ranging from low probability of overlap in blue, to a high probability of overlap in orange. Blue regions denotes a poor overlap in Normalised metric between the different parameters combinations (i.e. the parameters have a strong effect on the Normalised metric and thus the topological recovery). Conversely, orange regions denotes a high overlap in Normalised metric between the different parameters combinations (i.e. the parameters have a weak effect on the Normalised metric and thus the topological recovery)}.

\newpage
\begin{landscape}

\noindent
\textbf{Tables}\\
\bigskip

\noindent
\textbf{Table 1:} \hl{Bhattacharyya Coefficients of the pairwise method comparisons.}
\hl{Each comparisons corresponds to the comparison of the full (125) distributions between the ``best'' tree and the ``missing data'' trees for all the four methods (Maximum Likelihood; Bayesian consensus; Maximum Likelihood Bootstraps and Bayesian posterior trees) and the two Normalised Robins-Foulds (RF) and Triplets (Tr) metrics. Each line summarizes the distribution of the probability of overlap between pairs of tree inference methods. The values highlighted in bold are the extreme values of high or low probability of overlap between two methods. If two methods have a high probability of overlap, they have a similar ability to recover the ``correct" tree topology.}\\

\begin{table}[!ht]
%\begin{adjustwidth}{-2.25in}{0in} % Comment out/remove adjustwidth environment if table fits in text column.
\caption{}
\centering
\begin{tabular}{|l|c|c|c|c|c|c|c|}
  \hline
 Comparison &  Metric & Min. & 1st Qu. & Median & Mean & 3rd Qu. & Max. \\ 
  \hline
    Maximum Likelihood \textit{vs.} Bayesian consensus                 & $RF$ & \textbf{0.00} & \textbf{0.00} & 0.10 & 0.20 & 0.32 & \textbf{1.00} \\ 
                                                                       & $Tr$ & 0.34 & 0.49 & 0.61 & 0.62 & 0.75 & \textbf{1.00} \\ 
    Maximum Likelihood \textit{vs.} Maximum Likelihood bootstraps      & $RF$ & \textbf{0.03} & 0.54 & 0.69 & 0.64 & 0.77 & \textbf{0.98} \\ 
                                                                       & $Tr$ & 0.08 & 0.57 & 0.65 & 0.64 & 0.73 & 0.82 \\ 
    Maximum Likelihood \textit{vs.} Bayesian posterior trees           & $RF$ & \textbf{0.02} & 0.74 & 0.80 & 0.79 & 0.89 & \textbf{0.98} \\ 
                                                                       & $Tr$ & 0.21 & 0.67 & 0.73 & 0.72 & 0.77 & 0.84 \\ 
    Bayesian consensus \textit{vs.} Maximum Likelihood bootstraps      & $RF$ & \textbf{0.00} & \textbf{0.00} & \textbf{0.00} & \textbf{0.01} & \textbf{0.01} & \textbf{0.04} \\ 
                                                                       & $Tr$ & 0.08 & 0.38 & 0.59 & 0.57 & 0.73 & 0.84 \\ 
    Bayesian consensus \textit{vs.} Bayesian posterior trees           & $RF$ & \textbf{0.00} & \textbf{0.00} & \textbf{0.01} & \textbf{0.02} & \textbf{0.04} & 0.11 \\ 
                                                                       & $Tr$ & 0.21 & 0.36 & 0.56 & 0.55 & 0.74 & 0.87 \\ 
    Bayesian posterior tree \textit{vs.} Maximum Likelihood bootstraps & $RF$ & 0.50 & 0.77 & 0.85 & 0.85 & \textbf{0.96} & \textbf{1.00} \\ 
                                                                       & $Tr$ & 0.91 & \textbf{0.96} & \textbf{0.98} & \textbf{0.97} & \textbf{0.99} & \textbf{1.00} \\ 
   \hline
\end{tabular}
\label{Tab_Results-Difference_methods}
%\end{adjustwidth}
\end{table}
\end{landscape}

\end{document}
