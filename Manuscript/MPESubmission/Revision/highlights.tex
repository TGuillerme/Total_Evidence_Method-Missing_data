
\documentclass[12pt,letterpaper]{article}

%Packages
\usepackage{xcolor}
\usepackage{changepage}
\usepackage{pdflscape}
\usepackage{fixltx2e}
\usepackage{textcomp}
\usepackage{fullpage}
\usepackage{natbib}
\usepackage{float}
\usepackage{latexsym}
\usepackage{url}
\usepackage{epsfig}
\usepackage{graphicx}
\usepackage{amssymb}
\usepackage{amsmath}
\usepackage{bm}
\usepackage{array}
\usepackage[version=3]{mhchem}
\usepackage{ifthen}
\usepackage{caption}
\usepackage{hyperref}
\usepackage{amsthm}
\usepackage{amstext}
\usepackage{enumerate}
\usepackage[osf]{mathpazo}
\usepackage{dcolumn}
\usepackage{lineno}
\usepackage{longtable}
\pagenumbering{arabic}


%Pagination style and stuff
\linespread{2}
\raggedright
\setlength{\parindent}{0.5in}
%\setcounter{secnumdepth}{0} 

\begin{document}

\begin{enumerate}
\item{In the Total Evidence method, fossil placement rely only on morphological data.}
\item{Fossil placement is affect by the morphological data available for living taxa.}
\item{We suggest that the sampling effort should be focused on data for living taxa.}
\item{Also, Bayesian consensus trees are less affected by missing morphological data.}
\end{enumerate}


%We find that the number of living taxa with morphological characters and the overall number of morphological characters in the matrix, are more important than the amount of missing data in the fossil record for recovering the “best” tree topology. Therefore, we suggest that sampling effort should be focused on morphological data collection for living species to increase the accuracy of topological inference in a Total Evidence framework. Additionally, we find that Bayesian methods consistently outperform other tree inference methods. We therefore recommend using Bayesian consensus trees to fix the tree topology prior to further analyses.

\end{document}
