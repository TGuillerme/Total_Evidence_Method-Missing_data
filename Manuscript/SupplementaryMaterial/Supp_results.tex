\section{Supplementary material Section 3}

\begin{figure} 
\centering
    \includegraphics[width=1\textwidth]{SupplementaryMaterial/Supp_Figures/TEM_Fig-Appendix-Global-Bootstrap+Treeset-RF+Tr-small.pdf}
\caption{Trend of the effect of missing data on topological recovery on the Bootstraps and the Bayesian posterior trees distributions. The amount of missing data per parameter ($M_{L}$, $M_{F}$ and $M_{C}$) is represented along the x axis. The colour gradient from white to black represents respectively, 0\%, 10\%, 25\%, 50\% and 75\% of missing data. The topological recovery is represented on the y axis, both using Robinson-Foulds distance (upper row) and Triplets distance (lower row). Points represent the modal value of each distribution ; thick solid and thin dashed lines represents respectively the 50\% and 95\% confidence intervals or the distributions. The Bootstraps are represented in black and the Bayesian posterior trees distributions in grey.}
\label{Fig_global_BootTreesets} %Differences between all the parameters and between two methods (Boot vs treesets)
\end{figure}

\begin{figure} 
\centering
    \includegraphics[width=1\textwidth]{SupplementaryMaterial/Supp_Figures/TEM_Fig-Appendix-PairComp_permeth.pdf}
\caption{Distribution of the Pairwise Bhattacharyya Coefficients within each method. A- Distribution of the coefficients when comparing Ronbinson-Foulds distances. B- Distribution of the coefficients when comparing Triplets distances.}
\label{Fig_Bhatt.coeff_distribution} %Differences in overlap within each method
\end{figure}

\begin{figure} 
\centering
    \includegraphics[width=1\textwidth]{SupplementaryMaterial/Supp_Figures/TEM_Fig-Appendix-PairComp_within-Baytre-RF.pdf}
\caption{Pairwise Bhattacharyya Coefficients within the Bayesian posterior trees distributions. The pairwise trees comparisons are represent on both axis. The colour gradient from white to black represents respectively, 0\%, 10\%, 25\%, 50\% and 75\% of missing data. The matrix represents the values of pairwise Bhattacharyya Coefficients going from green (BC=1) to red (BC=0). Results are calculated for the Normalised Robinson-Foulds distance.}
\label{Fig_pairComp-Baytree-RF}
\end{figure} %Pairwise BC for the Bayesian treesets for RF. (parameters differences)

\begin{figure} 
\centering
    \includegraphics[width=1\textwidth]{SupplementaryMaterial/Supp_Figures/TEM_Fig-Appendix-PairComp_within-Baytre-Tr.pdf}
\caption{Pairwise Bhattacharyya Coefficients within the Bayesian posterior trees distributions. The pairwise trees comparisons are represent on both axis. The colour gradient from white to black represents respectively, 0\%, 10\%, 25\%, 50\% and 75\% of missing data. The matrix represents the values of pairwise Bhattacharyya Coefficients going from green (BC=1) to red (BC=0). Results are calculated for the Normalised Triplets distance.}
\label{Fig_pairComp-Baytree-Tr}
\end{figure} %Pairwise BC for the Bayesian treesets for Tr. (parameters differences)

\begin{figure} 
\centering
    \includegraphics[width=1\textwidth]{SupplementaryMaterial/Supp_Figures/TEM_Fig-Appendix-PairComp_within-MLbest-RF.pdf}
\caption{Pairwise Bhattacharyya Coefficients within the Maximum Likelihood trees. The pairwise trees comparisons are represent on both axis. The colour gradient from white to black represents respectively, 0\%, 10\%, 25\%, 50\% and 75\% of missing data. The matrix represents the values of pairwise Bhattacharyya Coefficients going from green (BC=1) to red (BC=0). Results are calculated for the Normalised Robinson-Foulds distance.}
\label{Fig_pairComp-MLbest-RF}
\end{figure} %Pairwise BC for the ML for RF. (parameters differences)

\begin{figure} 
\centering
    \includegraphics[width=1\textwidth]{SupplementaryMaterial/Supp_Figures/TEM_Fig-Appendix-PairComp_within-MLbest-Tr.pdf}
\caption{Pairwise Bhattacharyya Coefficients within the Maximum Likelihood trees. The pairwise trees comparisons are represent on both axis. The colour gradient from white to black represents respectively, 0\%, 10\%, 25\%, 50\% and 75\% of missing data. The matrix represents the values of pairwise Bhattacharyya Coefficients going from green (BC=1) to red (BC=0). Results are calculated for the Normalised Triplets distance.}
\label{Fig_pairComp-MLbest-Tr}
\end{figure} %Pairwise BC for the ML for Tr. (parameters differences)

\begin{figure} 
\centering
    \includegraphics[width=1\textwidth]{SupplementaryMaterial/Supp_Figures/TEM_Fig-Appendix-PairComp_within-MLboot-RF.pdf}
\caption{Pairwise Bhattacharyya Coefficients within the Bootstraps. The pairwise trees comparisons are represent on both axis. The colour gradient from white to black represents respectively, 0\%, 10\%, 25\%, 50\% and 75\% of missing data. The matrix represents the values of pairwise Bhattacharyya Coefficients going from green (BC=1) to red (BC=0). Results are calculated for the Normalised Robinson-Foulds distance.}
\label{Fig_pairComp-MLboot-RF}
\end{figure} %Pairwise BC for the Bootstraps for RF. (parameters differences)

\begin{figure} 
\centering
    \includegraphics[width=1\textwidth]{SupplementaryMaterial/Supp_Figures/TEM_Fig-Appendix-PairComp_within-MLboot-Tr.pdf}
\caption{Pairwise Bhattacharyya Coefficients within the Bootstraps. The pairwise trees comparisons are represent on both axis. The colour gradient from white to black represents respectively, 0\%, 10\%, 25\%, 50\% and 75\% of missing data. The matrix represents the values of pairwise Bhattacharyya Coefficients going from green (BC=1) to red (BC=0). Results are calculated for the Normalised Triplets distance.}
\label{Fig_pairComp-MLboot-Tr}
\end{figure} %Pairwise BC for the Bootstraps for Tr. (parameters differences)


\begin{table}[ht]
\centering
\begin{tabular}{rrrrrrr}
  \hline
 & Min. & 1st Qu. & Median & Mean & 3rd Qu. & Max. \\ 
  \hline
  Maximum likelihood-RF & 0.06 & 0.26 & 0.39 & 0.40 & 0.49 & 0.94 \\ 
  Maximum likelihood-Tr & 0.27 & 0.45 & 0.58 & 0.63 & 0.83 & 1.00 \\ 
  Bayesian consensus-RF & 0.70 & 0.71 & 0.73 & 0.76 & 0.79 & 0.96 \\ 
  Bayesian consensus-Tr & -0.25 & -0.09 & 0.18 & 0.21 & 0.40 & 0.98 \\ 
  Bootstraps-RF & 0.06 & 0.18 & 0.27 & 0.26 & 0.33 & 0.46 \\ 
  Bootstraps-Tr & 0.24 & 0.32 & 0.36 & 0.38 & 0.46 & 0.60 \\ 
  Bayesian posterior trees-RF & 0.16 & 0.22 & 0.32 & 0.34 & 0.42 & 0.65 \\ 
  Bayesian posterior trees-Tr & 0.24 & 0.36 & 0.40 & 0.50 & 0.68 & 0.98 \\ 
   \hline
\end{tabular}
\end{table}