\documentclass{article}
%called preambule, \documentclass[options]{command}, \usepackage, etc...
%example \documentclass[12pt]{article}
\usepackage{graphicx}
\usepackage{mathtools} %equationss
\usepackage{enumerate} %bullet points
%using packages: \usepackage[options]{name}


\begin{document}
\title{Combining living and fossil taxa into phylogenies: the missing data issue}
\date{Abstract}

\maketitle


Living species represent less than 1\% of all species that have ever lived. Ignoring fossil taxa may lead to misinterpretation of macroevolutionary patterns and processes such as trends in species richness, biogeographical history or paleoecology. This fact has led to an increasing consensus among scientists that fossil taxa must be included in macroevolutionary studies. One approach, known as the otal evidence method, uses molecular data from living taxa and morphological data from both living and fossil taxa to infer phylogenies. Although this approach seems very promising, it requires a lot of data. In particular it requires morphological data from both living and fossil taxa, both of which are scarce. Therefore, this approach is likely to suffer from having lots of missing data which may affect its ability to infer correct phylogenies.

Here we assess the effect of missing data on tree topologies inferred from total evidence supermatrices. Using simulations we investigate three major factors that directly affect the completeness of the morphological part of the supermatrix: (1) the proportion of living taxa with no morphological data, (2) the amount of missing data in the fossil taxa and (3) the overall number of morphological characters for all of the taxa.


\end{document}
