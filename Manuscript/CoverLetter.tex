\documentclass[a4paper,11pt]{article}

\usepackage[osf]{mathpazo}
\usepackage{lastpage} 
\pagenumbering{arabic}
\linespread{1.66}

\begin{document}

\begin{flushright}
Version dated: \today
\end{flushright}

\noindent{Dear Editors,}

%introducing paragraph
A major challenge in macroevolution since the advent of the molecular clock is to accurately estimate relationships among species through time. Today, two main approaches exist to investigate this issue: the cladistics method, focusing on palaeontological data and probabilistic methods focusing on neontological data. Both approaches have their benefits but also a major issue: they do not make use of all the available data (both molecular and morphological). A third emerging approach, the Total Evidence method, allows us to address this issue. This method is extremely promising since it allows us to study macroeveolutionary patterns and processes using all the available data. However, because of the amount of data involved, the Total Evidence method is likely to suffer from problems of missing data.

%What's our paper about
Our paper, entitled "Effect of missing data on topological inference using a total evidence approach", which we submit for your consideration, is, to our knowledge, the first thorough analysis to address the effect of missing data on recovering tree topology in a Total Evidence framework. We ran extensive computational analyses ($>$ 1.5 CPU century) %or 150 CPU years?
to provide insights into the effects of missing data in a Total Evidence framework. We test the effects of three missing data parameters (the number of living taxa with morphological data; the amount of missing data in the fossil record; and the overall number of missing morphological characters) on our ability to recover the "best" tree topology by using state-of-the-art methods in tree inference and comparisons.
%Our paper, titled “Ecology and mode-of-life explain lifespan variation in birds and mammals”, which we submit for your consideration, provides insight into the selection pressures that drive lifespan in endothermic vertebrates and gives novel evolutionary context to studies of species with exceptionally long lifespans. Our analysis of 1325 birds and mammals using novel analytical methods gives the most detailed and robust insight into correlates of lifespan to-date, allowing us to test many long-standing ecological and evolutionary hypotheses in a single analysis.

%What did we find
Our key findings are that two parameters: the number of living taxa with morphological data and the overall amount of morphological characters, are more important than the amount of missing data in the fossil record for recovering the "best" tree topology. Additionally, we found an effect of the tree inference method and we show that the Bayesian majority consensus tree always performs better than the Maximum Likelihood tree in recovering the "best" tree topology, regardless of the amount of missing data.

%What we suggest
Thus, we suggest that increasing the number of taxa with morphological data as well as the overall number of morphological characters are sufficient to improve the quality of Total Evidence tree topologies. Additionally, we suggest that fixing the topology in tree inference analysis using a Bayesian majority consensus tree could be used as a solid base for any additional analyses such as Tip-Dating or further Phylogenetic Comparative Methods.

We look forward to hearing from you soon.
\newline
\newline
\noindent{Sincerely,}
\newline
\newline
\noindent{Thomas Guillerme, on behalf of my co-author.}
\newline
\newline
Suggested reviewers: Liliana D\'{a}valos; David Pattinson; Fredrik Ronquist; Nicolas Salamin; Tanja Stadler.

\end{document}
